% Copyright (c) Utkarsh Patel
% Student, IIT Kharagpur

\documentclass[letterpaper,11pt]{article}

\usepackage{latexsym}
\usepackage[empty]{fullpage}
\usepackage{titlesec}
\usepackage{marvosym}
\usepackage[usenames,dvipsnames]{color}
\usepackage{verbatim}
\usepackage{enumitem}
\usepackage[pdftex]{hyperref}
\usepackage{fancyhdr}
\usepackage{fontawesome}
\usepackage{tipa}


\pagestyle{fancy}
\fancyhf{} % clear all header and footer fields
\fancyfoot{}
\renewcommand{\headrulewidth}{0pt}
\renewcommand{\footrulewidth}{0pt}

% Adjust margins
\addtolength{\oddsidemargin}{-0.375in}
\addtolength{\evensidemargin}{-0.375in}
\addtolength{\textwidth}{1in}
\addtolength{\topmargin}{-.5in}
\addtolength{\textheight}{1.0in}

\urlstyle{same}

\raggedbottom
\raggedright
\setlength{\tabcolsep}{0in}

% Sections formatting
\titleformat{\section}{
  \vspace{-4pt}\scshape\raggedright\large
}{}{0em}{}[\color{black}\titlerule \vspace{-5pt}]

%-------------------------
% Custom commands
\newcommand{\resumeItem}[2]{
  \item\small{
    \textbf{#1}{: #2 \vspace{-2pt}}
  }
}

\newcommand{\resumeSubheading}[4]{
  \vspace{-1pt}\item
    \begin{tabular*}{0.97\textwidth}{l@{\extracolsep{\fill}}r}
      \textbf{#1} & #2 \\
      \textit{\small#3} & \textit{\small #4} \\
    \end{tabular*}\vspace{-5pt}
}

\newcommand{\resumeSubItem}[2]{\resumeItem{#1}{#2}\vspace{-4pt}}

\renewcommand{\labelitemii}{$\circ$}

\newcommand{\resumeSubHeadingListStart}{\begin{itemize}[leftmargin=*]}
\newcommand{\resumeSubHeadingListEnd}{\end{itemize}}
\newcommand{\resumeItemListStart}{\begin{itemize}}
\newcommand{\resumeItemListEnd}{\end{itemize}\vspace{-5pt}}

%-------------------------------------------
%%%%%%  CV STARTS HERE  %%%%%%%%%%%%%%%%%%%%%%%%%%%%


\begin{document}

%----------HEADING-----------------
\begin{tabular*}{\textwidth}{l@{\extracolsep{\fill}}r}
  \textbf{\href {https://utkarsh512.github.io/}{\huge{Utkarsh Patel}}}\\ {3\textsuperscript{rd} Year Undergrad}& {{\faEnvelope}  \href{mailto:utkarshpatel@iitkgp.ac.in}{utkarshpatel@iitkgp.ac.in}}\\
  {\href{https://github.com/utkarsh512}{{\faGithub} utkarsh512}{  } \href{https://linkedin.com/in/utkarshiitkgp}{{\faLinkedinSquare} utkarshiitkgp}} & {{\faMobile} +91-95-4762-1111} \\
\end{tabular*}

%------------Interests----------------
\section{Interests}
  \resumeSubHeadingListStart
      {{Deep Learning \& Neural Networks} · {Algorithm Design} · {Graph Theory} · {Analog Circuit Design}}
  \resumeSubHeadingListEnd

%-----------EDUCATION-----------------
\section{Education}
  \resumeSubHeadingListStart

    \resumeSubheading
      {Indian Institute of Technology Kharagpur}{Kharagpur, India}
      {Candidate for Bachelor and Master of Technology (Dual Degree)}{Jul 2018 - Present}
      \resumeItemListStart
        \resumeItem{Major}
          {Electronics \& Electrical Communication Engineering | CGPA 9.54 / 10.0}
        \resumeItem{Minor}
          {Computer Science and Engineering | CGPA  10.0 / 10.0}
      \resumeItemListEnd
    \resumeSubheading
      {Shah Faiz Public School}{Ghazipur, India}
      {Central Board of Secondary Education}{}
      \resumeItemListStart
        \resumeItem{Higher Secondary}
          {94.8\% | May 2017}
        \resumeItem{Secondary}
          {CGPA 10 / 10 | May 2015}
      \resumeItemListEnd

    

  \resumeSubHeadingListEnd
  
\section{Research Experience}
  \resumeSubHeadingListStart

    \resumeSubheading
      {Functional Connectivity MRI Classification of Autism Spectrum Disorder \href{https://github.com/utkarsh512/fMRI-classification-of-ASD}{\faGithub}}{IIT Kharagpur}
      {Guide: \href{https://cse.iitkgp.ac.in/~dsamanta/}{Prof. Debasis Samanta}}{Aug 2020 - Present}
      \resumeItemListStart
        \resumeItem{Research Focus}
          {Application of machine learning algorithms to classify autism spectrum disorder (ASD) patients and typically developing (TD) participants.}
        \resumeItem{Data Collection}
          {Using \textbf{Resting-state functional MRI} (rs-fMRI) data from a large multisite data repository \textbf{ABIDE} (Autism Brain Imaging Data Exchange).}
        \resumeItem{Functional Brain Networks}
          {Using system-level graph analysis for evaluating brain networks  (default-mode, fronto-parietal, somatomotor, visual and cerebellar networks) and using functional connectivity analysis for extracting features.}
        \resumeItem{Model}
          {Identifying important features from machine learning algorithms and building and training a deep neural network for the classification problem.}
        \resumeItem{Testing}
          {Testing the deep neural network on examples of different age groups and different brain maps (CC400, CC200, AAL, HOA, TT, EZ, Dosenbach).}
      \resumeItemListEnd

  \resumeSubHeadingListEnd

  
\section{Course Projects}
  \resumeSubHeadingListStart

    \resumeSubheading
      {Cat or Not \href{https://github.com/utkarsh512/Image-Classification}{\faGithub}}{\href{https://www.deeplearning.ai/}{deeplearning.ai}}
      {Deep Learning application on Image Classification Problem}{Jul 2020}
      \begin{description}
        \itemsep-0.25em
        \item[$\circ$] {Built a deep neural network to classify images as a cat image or a non-cat image.}
        \item[$\circ$] {Coded the \textbf{Forward Propagation} and the \textbf{Backward Propagation} from scratch to train the model. }
        \item[$\circ$] {Used \textbf{Batch Gradient Descent} algorithm to get optimal weights and biases.}
        \item[$\circ$] {Achieved \textbf{80\%} accuracy on test set after training the model.}
    \end{description}
    

  \resumeSubHeadingListEnd


  
 \section{Relevant Coursework}
  \resumeSubHeadingListStart
    \resumeSubItem{Computer Science}
      {\\Algorithms (+ lab), Programming and Data Structures (+ lab)}
    \resumeSubItem{Deep Learning}
      {\\Natural Language Processing*, Regularization \& Optimization Techniques*, \href{https://www.coursera.org/account/accomplishments/certificate/9HS82Y4DP4DV}{Neural Networks and Deep Learning}}
    \resumeSubItem{Electronics and Communication Engineering}
      {\\Digital Electronics (+ lab)*, Analog Communication (+ lab)*, RF \& Microwave (+ lab)*, Digital Speech Processing, Analog Electronics (+ lab), Control Theory*, Signals \& Systems, Semiconductor Devices (+ lab)}
    \resumeSubItem{Mathematics}
      {\\Graph Theory, Probability and Stochastic Processes, Matrix Algebra}
    \rightline{* \textit{denotes ongoing courses}}
  \resumeSubHeadingListEnd


%-----------EXPERIENCE-----------------
\iffalse
\section{Experience}
  \resumeSubHeadingListStart

    \resumeSubheading
      {Google}{Mountain View, CA}
      {Software Engineer}{Oct 2016 - Present}
      \resumeItemListStart
        \resumeItem{Tensorflow}
          {TensorFlow is an open source software library for numerical computation using data flow graphs; primarily used for training deep learning models.}
        \resumeItem{Apache Beam}
          {Apache Beam is a unified model for defining both batch and streaming data-parallel processing pipelines, as well as a set of language-specific SDKs for constructing pipelines and runners.}
      \resumeItemListEnd

    \resumeSubheading
      {Coursera}{Mountain View, CA}
      {Senior Software Engineer}{Jan 2014 - Oct 2016}
      \resumeItemListStart
        \resumeItem{Notifications}
          {Service for sending email, push and in-app notifications. Involved in features such as delivery time optimization, tracking, queuing and A/B testing. Built an internal app to run batch campaigns for marketing etc.}
        \resumeItem{Nostos}
          {Bulk data processing and injection service from Hadoop to Cassandra and provides a thin REST layer on top for serving offline computed data online.}
        \resumeItem{Workflows}
          {Dataduct an open source workflow framework to create and manage data pipelines leveraging reusables patterns to expedite developer productivity.}
        \resumeItem{Data Collection}
          {Designed the internal survey and crowd sourcing platfowm which allowed for creating various tasks for crowd sourding or embedding surveys across the Coursera platform.}
        \resumeItem{Dev Environment}
          {Analytics environment based on docker and AWS, standardized the python and R dependencies. Wrote the core libraries that are shared by all data scientists.}
        \resumeItem{Data Warehousing}
          {Setup, schema design and management of Amazon Redshift. Built an internal app for access to the data using a web interface. Dataduct integration for daily ETL injection into Redshift.}
        \resumeItem{Recommendations}
          {Core service for all recommendation systems at Coursesa, currently used on the homepage and throughout the content discovery process. Worked on both offline training and online serving.}
        \resumeItem{Content Discovery}
          {Improved content discovery by building a new onboarding experience on coursera. Using this to personalize the search and browse experience. Also worked on ranking and indexing improvements.}
        \resumeItem{Course Dashboards}
          {Instructor dashboards and learner surveying tools, which helped instructors run their class better by providing data on Assignments and Learner Activity.}
      \resumeItemListEnd

    \resumeSubheading
      {Lucena Research}{Atlanta, GA}
      {Data Scientist}{Summer 2012 and 2013}
      \resumeItemListStart
        \resumeItem{Portfolio Management}
          {Created models for portfolio hedging,  portfolio optimization and price forecasting. Also creating a strategy backtesting engine used for simulating and backtesting strategies.}
        \resumeItem{QuantDesk}
          {Python backend for a web application used by hedge fund managers for portfolio management.}
      \resumeItemListEnd

    \resumeSubheading
      {Georgia Institute of Technology}{Atlanta, CA}
      {Research and Teaching Assistant}{Jan 2012 - Dec 2013}
      \resumeItemListStart
        \resumeItem{Research Assistant - Machine Learning}
          {Research on machine learning for portfolio hedging and replication algorithms. Modeling low-risk \& continuous-return strategies. Developed the python library QSTK.}
        \resumeItem{Teaching Assistant - Computational Investing}
          {The online course on Coursera, had more than 100,000 students enrolled. It was featured on the 11 Alive News and the Atlanta Journal Constitution. Involved in creating assignment, exams and conducting recitation sessions. Also taught the on-campus version of the course.}
      \resumeItemListEnd

  \resumeSubHeadingListEnd

\fi 

%-----------PROJECTS-----------------
\section{Scholastic Achievements}
  \resumeSubHeadingListStart
      \itemsep-0.5em
      \item{2020: Holding \textbf{Department rank 1} among 53 dual degree students at the end of 4\textsuperscript{th} semester.}
      \item{2017: Secured \textbf{2\textsuperscript{nd} position} in the district in All India Senior School Certificate Examination.}
      
  \resumeSubHeadingListEnd
\section{Technical Skills}
  \resumeSubHeadingListStart
    \resumeSubItem{Programming Languages}
      {\\Python, C/C++, Octave, MySQL}
    \resumeSubItem{Libraries / Frameworks}
      {\\TensorFlow, PyTorch, sklearn, Pandas, NumPy, MatplotLib, PIL, SciPy, C++ STL}
    \resumeSubItem{Softwares / Platforms / OS}
      {\\Google Cloud, MATLAB, LTSpice, Jupyter, Git, \LaTeX, Windows, Ubuntu}
    \resumeSubItem{Competitive Programming}
      {\\\href{https://codeforces.com/profile/QMark}{CodeForces}}
  \resumeSubHeadingListEnd
  


%
%--------PROGRAMMING SKILLS------------
%\section{Programming Skills}
%  \resumeSubHeadingListStart
%    \item{
%      \textbf{Languages}{: Scala, Python, Javascript, C++, SQL, Java}
%      \hfill
%      \textbf{Technologies}{: AWS, Play, React, Kafka, GCE}
%    }
%  \resumeSubHeadingListEnd


%-------------------------------------------
\end{document}
